\documentclass[12pt]{article}
\usepackage[tagged, highstructure]{accessibility}
\usepackage[english]{babel}
\usepackage[utf8x]{inputenc}
\usepackage[T1]{fontenc}
\usepackage[margin=1in]{geometry}
\usepackage{scribe}
\usepackage{listings}
\usepackage{natbib,verbatim}
\usepackage{amsmath,amssymb,amsfonts,mathtools}
\renewcommand{\E}{\mathbb{E}}
\usepackage{hyperref}
\hypersetup{
    colorlinks=true,
    linkcolor=blue,
    filecolor=magenta,      
    urlcolor=magenta,
    pdftitle={Homework 1},
    pdfauthor={Nisha Chandramoorthy},
    pdflang={en-US}
}

%\Scribe{Your Name}
\title{Homework 1_31310}
\LectureNumber{CAAM 31310}
\LectureDate{Due Oct 15, '24 (11:59 pm ET) on Gradescope} 
\Lecturer{Cite any sources and collaborators; do not copy. See syllabus for policy.}
\LectureTitle{Homework 1}

\lstset{style=mystyle}

\begin{document}
\MakeScribeTop

\section*{Problem 1}
Take $X_i$ to be independent zero-mean random variables with unit variance.
Usually, the classical central limit theorem (CLT) is proved by showing that the characteristic 
function of $Y_n := (1/\sqrt{n}) \sum_{i=1}^n X_i$ converges to the \href{}{characteristic function} of a 
standard normal. 
Our task in this problem is to use \href{}{Stein's identity} to prove the CLT. Stein's method has been used 
to derive computational methods for bounding the distance between two random variables and for a class 
of sampling algorithms, where the task is to generate samples from a partially specified probability density.
\begin{itemize}
	\item[(1)] First show that if $W$ is a standard normal RV, for any bounded, differentiable function $f:\mathbb{R} \to \mathbb{{R}},$ with $\E [W f(W)], \E f'(W) \leq \infty,$
				$\mathcal{A}f (W) := f'(W) -  W f(W)$ has zero mean (1 point). This is known as \emph{Stein's lemma}.
	\item[(2)] For any function $g \in {\rm Lip}_1(\mathbb{R})$ (differentiable functions with Lipschitz constant = 1), there is a solution $f$ 
				 to \emph{Stein's equation}: $\mathcal{A} f (x) = g(x) - \E g(W),$ where $W$ is a standard normal. (2 points)
	\item[(3)] Show that a bounded solution $f$ exists that is twice-differentiable with $\|f''\|_\infty \leq 2$ and $\|f'\|_\infty \leq \sqrt{\pi}/2.$ (2 points)
	\item[(4)] Let $\E|X_i|^3 \leq \infty.$ Using (2) and (3) above, find a function class $\mathcal{F}$ such that the Wasserstein-1 distance,
	 $$ W^1(Y_n, W) := \sup_{g \in {\rm Lip}_1(\mathbb{R})} |\E g(W) - \E g(Y)| \leq \sup_{f \in \mathcal{F}} |\E\mathcal{A} f(Y_n)| \leq C \E|X_i^3|/\sqrt{n}. $$ (2 points)
	\item[(5)] Use (4) and the converse of Stein's lemma to prove the CLT: $Y_n \xrightarrow[]{\rm d} W.$ (2 points) 
\end{itemize}

\section*{Problem 2}
This problem asks you to think about an iterative numerical method as a discrete-time dynamical system (map). Consider the power iteration method for a square, non-singular, diagonalizable matrix $A \in \mathbb{R}^{d\times d}.$ For $t \in \mathbb{N},$
\begin{itemize}
	\item $v_t \to Av_{t-1}$
	\item[(*)] $v_{t+1} \to v_{t+1}/\|v_{t+1}\|.$
\end{itemize}
\begin{enumerate}
	\item Write down a map $F(x_t) = x_{t+1}$ to describe the above algorithm, where $F$ is defined on a set $M \subseteq \mathbb{R}^d$. (1 point)
	\item Is $M$ compact? (1 point)
	\item Is $F$ a contraction on $M$? (1 point)
	\item How many fixed points does $F$ have? (1 point) What are they? (1 point)
	\item State the assumptions on $A$ so that almost every initial condition converges to a fixed point. (1 point)
	\item Under the assumptions in the part above, prove the convergence of almost every iterate to a fixed point of $F.$ (3 points)
	\item From here on, consider the power iteration without the normalization step (*). Write the corresponding new map, $F$, on $\mathbb{R}^d$ (1 point).
	\item Give conditions on $A$ for $F$ to be a contraction map (1 point).
	\item Give conditions on $A$ for $F$ to be a linear hyperbolic map (1 point).
	\item Without the additional conditions in the above two parts (i.e., without hyperbolicity assumptions), describe the asymptotic behavior of all orbits of $F.$ That is, give, with justification, a stable-unstable-center decomposition of $\mathbb{R}^d$ by $F.$ (3 points).

\end{enumerate}


\section*{Problem 3}
The dataset you are given consists of $d$-dimensional embeddings of $n$ sentences from (random articles of) simple English Wikipedia.
These embeddings, referred to as $ET$, are calculated with the sentence transformer model from \href{https://huggingface.co/mixedbread-ai/mxbai-embed-large-v1}{here}. 

Consider the following alternative way, referred to as $ESVD$, to obtain word embeddings. Construct a matrix, $A$, of size $n \times n_c$ where 
$n_c$ is the number of some chosen ``context'' words found in the data. An entry $A_{ij}$ is set to 1 if the $i$th sentence contains the $j$th context word. 
From the best rank $d$ approximation of $A$ (its SVD), find an embedding for each of the $n$ sentences.
You may choose, say the top $n_c$ words (measured by their overall frequency of occurrence) as your context words. Justify any alternative choice for the context words 
in your answers for the questions below.
\begin{itemize}
\item[(1)] For $d = 2,$ plot a histogram of your sentence embeddings from $ET$ and $ESVD$ (1 point). Explain your observations, and particularly the dependence of the distribution 
 you see on $n_c$ and on the embedding method (using an SVD or transformers). (3 points)
\item[(2)] Check numerically if the classical CLT holds for $d=2$. Justify your answer with an appropriate convergence plot, e.g., Wasserstein distance between the scaled sample mean and a Gaussian vs no. of samples. (2 points)
\item[(3)] As $d$ and $n$ increase, $ESVD$ can produce more identically distributed embeddings. Justify why or not (1 point)  
\end{itemize}


\end{document}
