\documentclass[12pt]{article}
\usepackage[tagged, highstructure]{accessibility}
\usepackage[english]{babel}
\usepackage[utf8x]{inputenc}
\usepackage[T1]{fontenc}
\usepackage[margin=1in]{geometry}
\usepackage{scribe}
\usepackage{listings}
\usepackage{natbib,verbatim}
\usepackage{amsmath,amssymb,amsfonts,mathtools}
\renewcommand{\E}{\mathbb{E}}
\usepackage{hyperref}
\hypersetup{
    colorlinks=true,
    linkcolor=blue,
    filecolor=magenta,      
    urlcolor=magenta,
    pdftitle={Homework 2},
    pdfauthor={Nisha Chandramoorthy},
    pdflang={en-US}
}

%\Scribe{Your Name}
\title{Homework 3_31310}
\LectureNumber{CAAM 31310}
\LectureDate{Due Nov 9th, '24 (11:59 pm ET) on Gradescope} 
\Lecturer{Cite any sources and collaborators; do not copy. See syllabus for policy.}
\LectureTitle{Homework 3}

\lstset{style=mystyle}

\begin{document}
\MakeScribeTop
In this homework, we will study a gradient flow on $\mathbb{T}^2.$ For $w = [x,y,z]\in \mathbb{R}^3,$ define $\ell(w) = x(w)$ to be the first coordinate function. 
The gradient flow, $F^t,$ of a function $w \to \ell(w)$ is given by $d F^t(w)/dt = -\nabla F(w),$ where $\nabla$ here is the gradient operator induced by the Euclidean metric on $\mathbb{R}^2.$ 
For concreteness, our torus will be as shown below, with $\max_{w} x(w) = \max_w y(w) = 10 + 1 = 11$ and $\max_w z(w) = 1.$ You can play with the script \verb+torus_gradientflow.py+ for simulation and visualization of the flow.
\begin{figure}[h!]
    \includegraphics[width=0.5\textwidth]{torus.png}
    \caption{A torus embedded in $\mathbb{R}^3.$}
\end{figure}
\begin{itemize}
    \item Give all the fixed points of $F^t.$ (1 point)
    \item Is $w\to \ell(w)$ geodesically convex on $\mathbb{T}^2$? (1 point) 
    \item Is $w \to x(w) + 11$ a Lyapunov function? If yes, define an appropriate neighborhood around the fixed points for its definition. (3 points)
    \item Using the Lyapunov function defined above or otherwise, prove the asymptotic stability of all orbits to the set of fixed points. (3 points)
    \item Change the embedding of $\mathbb{T}^2,$ by applying a rotation of the $x$-$y$ plane by an $\alpha > 0.$ How must the Lyapunov function be modified? (2 points)
    \item Write down a sum of squares optimization problem using polynomial functions that recovers the Lyapunov function for any $\alpha.$ You do not need to submit a code, only the formulation. (3 points) 
\end{itemize}
\end{document}
