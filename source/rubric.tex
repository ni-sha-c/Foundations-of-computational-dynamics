\documentclass[12pt]{article}
\usepackage[tagged, highstructure]{accessibility}
\usepackage[english]{babel}
\usepackage[utf8x]{inputenc}
\usepackage[T1]{fontenc}
\usepackage[margin=1in]{geometry}
\usepackage{scribe}
\usepackage{listings}
\usepackage{natbib,verbatim}
\usepackage{hyperref}
\hypersetup{
    colorlinks=true,
    linkcolor=blue,
    filecolor=magenta,      
    urlcolor=magenta,
    pdftitle={Course Syllabus},
    pdfauthor={Nisha Chandramoorthy},
    pdflang={en-US}
}

%\Scribe{Your Name}
\title{Syllabus-31310}
\Lecturer{Nisha Chandramoorthy (nishac@uchicago.edu)}
\LectureNumber{\begin{large} Foundations of Computational Dynamics\end{large}}
\LectureDate{Dec 2024}
\LectureTitle{\begin{large}CAAM/STAT 31310\end{large}}

\lstset{style=mystyle}

\begin{document}
\MakeScribeTop

\begin{itemize}
	\item You are free to delve deeper into any subset of topics covered in class, or more
	broadly, into mathematical analyses and computational aspects of dynamical systems. You are actively encouraged
	 to find topics that are at the intersection of your research with the course
	material.
	\item The project will be evaluated on its computational and theoretical components. You
	can maximize your score by showing substantial work (2-3 homeworks worth) in
	the algorithmic/computational aspects or the theoretical aspects or both.
	\item A self-contained final report of no more than 5 pages is required at the end of term.
	This carries 65\% of the final project score.
	\item A 10-minute presentation, including questions, briefly summarizing project components is also required.
	This carries 25\% of the final project score. Sign-up sheets for this are \href{https://docs.google.com/document/d/1DLgG3xIfzimgwl_O6E8mrKifF1pC9hNf6Y7dAF18Cqc/edit?usp=sharing}{here}.
	\item Your report can follow your proposal outline but elaborate on the problem
	motivation (what problem are you trying to solve or understand), your proposed
	solution approach, and your inferences (what you have learned).
	\item  The final report will be evaluated based on your formulation/modeling of the problem in mathematical terms, your solution strategy and your discussion on your
	choice of solution strategy. You will not be evaluated on how novel your solution is; only how well you have understood your problem and solution method,
	and how clearly your justify (using numerical analysis and dynamical systems theory taught in class)
	your results.
	\item Computational questions you answer may include, but not be limited to, 
	i) data assimilation and inverse problems, ii) dimension reduction and feature extraction, iii) learning dynamical systems, iv) uncertainty quantification in parameters/states of dynamical systems, 
	and v) perturbation/stability analyses of dynamical systems.
\end{itemize}

%%%%%%%%%%% end of doc
\end{document}
