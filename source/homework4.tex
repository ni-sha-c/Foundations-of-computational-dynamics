\documentclass[12pt]{article}
\usepackage[tagged, highstructure]{accessibility}
\usepackage[english]{babel}
\usepackage[utf8x]{inputenc}
\usepackage[T1]{fontenc}
\usepackage[margin=1in]{geometry}
\usepackage{scribe}
\usepackage{listings}
\usepackage{natbib,verbatim}
\usepackage{amsmath,amssymb,amsfonts,mathtools}
\renewcommand{\E}{\mathbb{E}}
\usepackage{hyperref}
\hypersetup{
    colorlinks=true,
    linkcolor=blue,
    filecolor=magenta,      
    urlcolor=magenta,
    pdftitle={Homework 4},
    pdfauthor={Nisha Chandramoorthy},
    pdflang={en-US}
}

%\Scribe{Your Name}
\title{Homework 4_31310}
\LectureNumber{CAAM 31310}
\LectureDate{Due Nov 24th, '24 (11:59 pm ET) on Gradescope} 
\Lecturer{Cite any sources and collaborators; do not copy. See syllabus for policy.}
\LectureTitle{Homework 4}

\lstset{style=mystyle}

\begin{document}
\MakeScribeTop
In this homework, we will explore Lyapunov analysis and computation of Lyapunov vectors and exponents. We will consider the three-variable Lorenz '63 system, which was introduced as a reduced order model for atmospheric convection. Let $x = [a,b,c]$ be a phase point, and $a(x)$ refers to the first component/coordinate at $x.$ 
The system is given by the ODEs:
\begin{equation}
	\dfrac{d\varphi^t(x)}{dt} = v(\varphi^t(x)) = 
    \begin{bmatrix}
	\sigma(b - a) \\
	a(\rho - c) - b \\
	ab  - \beta c
	\end{bmatrix} \circ \varphi^t(x).
\end{equation}
Fix $\sigma = 10, \beta = 8/3$ and $\rho = 14$. 
\begin{enumerate}
	\item[I] Enumerate all the fixed point attractors and indicate their stability. (2 points) 
	\item[II] (3 points) Compute the three Lyapunov exponents. Submit a code snippet, explaining each line. Do they depend on the initial condition?
    \item[III] (3 points) Use the linearized system around a fixed point to define a Lyapunov function (i.e., check that your definition satisfies the properties of a Lyapunov function).
    \item[IV] (3 points) Write down a sum-of-squares optimization problem (you don't need to solve it) for the region of stability of a fixed point.  
	\item[III] (5 points) Are the following statements true or false for the Lorenz '63 system? Provide justification.
		\begin{enumerate}
			\item The adjoint covariant Lyapounov vector is the same as the covariant Lyapunov vector for the top LE
			\item The top (backward) Lyapunov vector is always covariant
			\item There is a zero Lyapunov exponent for the ODE system
			\item The stable adjoint Lyapunov vector is perpendicular to the unstable Lyapunov vector 
			\item There is a dense set of points on the attractor that result in different LEs than the ones computed
		\end{enumerate}
	\item[IV] (5 points) Compute and plot the top adjoint Lyapunov vector. Submit a code snippet/algorithm.

\end{enumerate} 
\end{document}
